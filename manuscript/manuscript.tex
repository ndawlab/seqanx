\documentclass[11pt]{article} % For LaTeX2e
\usepackage{manuscript,palatino}
\usepackage{graphicx}
\usepackage{amsfonts,amsmath}
\usepackage{algorithm, algpseudocode}%

\title{Anxiety: a decision-theoretic perspective}

\author{
Samuel Zorowitz \\
Princeton Neuroscience Institute\\
Princeton University\\
Princeton, NJ 08540 \\
\texttt{zorowitz@princeton.edu} \\
\And
Nathaniel Daw \\
Princeton Neuroscience Institute\\
Princeton University\\
Princeton, NJ 08540 \\
\texttt{ndaw@princeton.edu} \\
}

% The \author macro works with any number of authors. There are two commands
% used to separate the names and addresses of multiple authors: \And and \AND.
%
% Using \And between authors leaves it to \LaTeX{} to determine where to break
% the lines. Using \AND forces a linebreak at that point. So, if \LaTeX{}
% puts 3 of 4 authors names on the first line, and the last on the second
% line, try using \AND instead of \And before the third author name.

\newcommand{\fix}{\marginpar{FIX}}
\newcommand{\new}{\marginpar{NEW}}

\begin{document}

\maketitle

\begin{abstract}
To be filled in
\end{abstract}

\keywords{
reinforcement learning; avoidance; anxiety
}

\startmain % to start the main 1-4 pages of the submission.

\section{Introduction}


\section{Methods}

\subsection{Markov Decision Processes}

Here, we briefly review the formalism of Markov decision processes (MDPs), which
provide the foundation for our results. For more complete treatments of the
subject, see Sutton \& Barto (1998, 2018) and Bertsekas \& Tsitsiklis (1996).

A MDP is defined by a set of states, $S$, a set of actions, $A$, a reward
function, $R(s,a)$ over state/action pairs, and a state transition distribution,
$P(s'|s,a)$, where $a$ denotes a chosen action. States and rewards occur
sequentially according to these one-step functions, driven by a series of
actions; the goal is to learn to choose a probabilistic policy over actions,
denoted by $\pi$, that maximizes the value function, $V^\pi(s)$, defined as the
expected cumulative discounted reward:

$$ V^\pi(s) = \mathbb{E} \left[ \sum^{\infty}_{k=0} \gamma^k R_{t+k} | S_t = s \right] $$

Here, $\gamma$ is a parameter controlling temporal discounting. The value function
can also be defined recursively as the sum of the immediate reward of the action
chosen in that state, $R(s, a)$, and the value of its successor state $s’$,
averaged over possible actions, $a$, and transitions that would occur if the agent
chose according to $\pi$:

$$ V^\pi(s) = \sum_a \pi(a|s) \left[ R(s,a) + \sum_{s'} P(s'|s,a) \gamma V^\pi (S') \right] $$

The value function under the optimal policy is given by:

$$ V^*(s) = \max_a \mathbb{E} \left[ \sum^{\infty}_{k=0} \gamma^k R_{t+k} | S_t = s \right] $$

$$ = \max_a \left[ R(s,a) + \sum_{s'} P(s'|s,a) \gamma V^\pi (S') \right] $$

Knowledge of the value function can help to guide choices. For instance, we can
define the state-action value function as the value of choosing action $a$ and
following $\pi$ thereafter:

$$ Q^\pi(s,a) = \mathbb{E} \left[ \sum^{\infty}_{k=0} \gamma^k R_{t+k} | S_t = s \right] $$

$$ = R(s,a) + \sum_{s'} P(s'|s,a) \gamma V^\pi (S') $$

Then at any state one could choose the action that maximizes $Q^\pi(s,a)$. Formally
this defines a new policy, which is as good or better than the baseline policy
$\pi$; analogously, Eq 2 can be used to define the optimal state-action value
function, the maximization of which selects optimal actions. Note that it is
possible to write a recursive definition for $Q$ in the same manner as Eq 1, and
work directly with the state-action values, rather than deriving them indirectly
from $V$.

\subsection{Robust Control}

To create the symptoms of anxiety, we need a loss of control. This can come in
two forms. First is through the state transition distribution, $P(s'|s,a)$.
Normally, we assume the world is deterministic; in other words, an agent's chosen
action, $a$ leads the agent to the desired successor state, $s'$. In a stochastic
environment, however, an agent may instead transition to an undesired state, $s^*$.

Alternately, an agent may choose optimally in the present but may be uncertain of
its choice control in the future. In such a case, we deviate from the optimal
policy in the future. We can adopt a different function. There are a number to
choose from (Garcia \& Fernandez, 2015). Here we adopt the beta-pessimism criterion
(Gaskett, 2003):

$$ V(s') = \beta \max_{a'} Q(s',a') + (1 - \beta) \min_{a'} Q(s',a') $$

where $\beta$ controls the degree of pessimism. When $\beta = 1$, an agent expects
complete control to choose the best action (best-case criterion); when $\beta = 0$,
an agent expects the  worst future case; and when $\beta = 0.5$, an agent expects
randomness. (We are not committed to this particular instantiation; it is simply
a convenient notation. Other parameterizations work equally well, see appendix.)

\subsection{Simulations}

In some instances, we solve for Q-values and state values directly through value
iteration. This is a form of Dynamic Programming. In other instances, we estimate
these terms through temporal difference learning. We use beta-pessimism (betamax)
learning for both. We use the former to solve for final (infinite horizon) values
and the latter to identify finite horizon values. We explain when we use both.

\section{Results}

\subsection{Pessimistic Inference}

First and foremost, pathological anxiety is associated with pessimistic expectations
about the probability and cost of future negative events (Grupe \& Nitschke, 2013;
Bishop \& Gagne, 2018). In fear conditioning, individuals with anxiety are more
likely to exhibit increased threat appraisal for safety signals (CS-) and the
extinguished fear cue (CS+) (Lissek et al. 2005; Duits et al., 2015). In considering
hypothetical future negative outcomes, individuals with anxiety are more likely
to evaluate negative outcomes as more likely and their consequences more severe
(pick a few references).

Importantly, in the clinical literature, pathological anxiety is also associated
with a lack of perceived control (Gallagher et al., 2014). Studies suggest that
this belief mediates anxiety (ref?). This can come in whatever form.

In the first series of simulations, we show directly how lack of perceived control
can directly account for pessimistic inference. In the open field environment
(Figure 1a), an agent navigates a simple grid-world. The world consists of almost
entirely of empty tiles, except for two which yield a reward and punishment,
respectively.

Under complete control (objective or perceived), an agent in this environment is
under no threat despite the presence of an aversive state. This is simply because,
in a full controllable world, a (non-masochistic) agent is never forced to interact
with the aversive state. When this control breaks down -- through a stochastic
world or a distrust of future competency -- an agent can no longer guarantee it
will never encounter the aversive state. In such a scenario, the negative value
back-propagates to its predecessor states through the recursive formulas defined
above. The result is an environment that appears worse (i.e. less positive) than
it otherwise would.

We shoe this above. Using value iteration. Show it for best-case scenario. Show
it for expected random. Show it for worst-case scenario.

\begin{figure}
  \centerline{%
    \resizebox{1.0\textwidth}{!}{\includegraphics[trim={0 0 0 0},clip]{../figures/01_field.png}}%
  }
  \caption{\textbf{Open Field}}
  \par Oooh so pretty.
\end{figure}

\subsection{Avoidance}

\begin{figure}
  \centerline{%
    \resizebox{1.0\textwidth}{!}{\includegraphics[trim={0 0 0 0},clip]{../figures/02_cliff.png}}%
  }
  \caption{\textbf{Cliff Walking}}
  \par You better watch out YOU BETTER WATCH OUT
\end{figure}

\subsection{Approach-Avoidance Conflict}

\subsection{Planning}

\subsection{Free Choice Premium}

\subsection{Longitudinal Progression}


\section{Appendix}
\subsection{Algorithms}

\begin{algorithm}
  \caption{Value Iteration}

  \State Algorithm parameter: a small threshold $\theta > 0$ determining accuracy of estimation
  \State Initialize $V(s)$, for all $s \in S$ arbitrarily, except that $V(terminal) = 0$
  \State
  \While{$\Delta > \theta$}
    \State $\Delta \leftarrow 0$
    \Loop \ for each $s \in S$
      \State $v \leftarrow V(s)$
      \State $ V(s) = \max_a \sum_{s',r} p(s',r|s,a) \left[ r + \gamma V(s') \right] $
      \State $\Delta \leftarrow \max(\Delta, |v - V(s)|)$
    \EndLoop
  \EndWhile

\end{algorithm}

\begin{algorithm}
  \caption{Betamax Temporal Difference Learning}

  \State Algorithm parameters: step size $\eta \in (0, 1)$, small $\epsilon > 0$
  \State Initialize $Q(s,a)$, for all $s \in S$, $a \in A$ arbitrarily, except that $Q(terminal,\cdot) = 0$
  \State
  \Loop \ for each episode
    \State Initialize $S$
    \While{$S \notin S(terminal)$}
      \State Choose $A$ from $S$ using policy derived from $Q$
      \State Take action $A$, observe $R, S$
      \State $Q(S, A) \leftarrow Q(S, A) + \eta \left[ R + \gamma \max_a Q(S',a) − Q(S, A) \right] $
      \State $S \leftarrow S'$
    \EndWhile
  \EndLoop

\end{algorithm}

\end{document}
